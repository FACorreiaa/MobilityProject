\documentclass[11pt]{report}
\renewcommand{\baselinestretch}{1.6}

\usepackage[T1]{fontenc} 
\usepackage[utf8]{inputenc}
\usepackage{graphicx} 
\usepackage[portuguese]{babel} 
\usepackage{hyphenat}
\usepackage[affil-it]{authblk} 
\usepackage{amsmath}
\usepackage{hyperref}
\usepackage{float}
\usepackage{enumerate}


\usepackage[utf8]{inputenc}
\usepackage[T1]{fontenc}
\usepackage{tabularx,booktabs,ragged2e}
\newcolumntype{L}{>{\RaggedRight\arraybackslash}X} % modified 'X' col. type
\usepackage[skip=0.333\baselineskip]{caption} % optional
\usepackage{multirow}

\usepackage{color}
\definecolor{lightgray}{rgb}{0.95, 0.95, 0.95}
\definecolor{darkgray}{rgb}{0.4, 0.4, 0.4}
%\definecolor{purple}{rgb}{0.65, 0.12, 0.82}
\definecolor{editorGray}{rgb}{0.95, 0.95, 0.95}
\definecolor{editorOcher}{rgb}{1, 0.5, 0} % #FF7F00 -> rgb(239, 169, 0)
\definecolor{editorGreen}{rgb}{0, 0.5, 0} % #007C00 -> rgb(0, 124, 0)
\definecolor{orange}{rgb}{1,0.45,0.13}		
\definecolor{olive}{rgb}{0.17,0.59,0.20}
\definecolor{brown}{rgb}{0.69,0.31,0.31}
\definecolor{purple}{rgb}{0.38,0.18,0.81}

\definecolor{lightblue}{rgb}{0.1,0.57,0.7}
\definecolor{lightred}{rgb}{1,0.4,0.5}
\usepackage{upquote}
\usepackage{listings}

% JavaScript
\lstdefinelanguage{JavaScript}{
  morekeywords={typeof, new, true, false, catch, function, return, null, catch, switch, var, if, in, while, do, else, case, break},
  morecomment=[s]{/*}{*/},
  morecomment=[l]//,
  morestring=[b]",
  morestring=[b]'
}


\lstdefinestyle{htmlcssjs} {%
  % General design
%  backgroundcolor=\color{editorGray},
  basicstyle={\footnotesize\ttfamily},   
  frame=b,
  % line-numbers
  xleftmargin={0.75cm},
  numbers=left,
  stepnumber=1,
  firstnumber=1,
  numberfirstline=true,	
  % Code design
  identifierstyle=\color{black},
  keywordstyle=\color{blue}\bfseries,
  ndkeywordstyle=\color{editorGreen}\bfseries,
  stringstyle=\color{editorOcher}\ttfamily,
  commentstyle=\color{brown}\ttfamily,
  % Code
  language=JavaScript,
  alsodigit={.:;},	
  tabsize=2,
  showtabs=false,
  showspaces=false,
  showstringspaces=false,
  extendedchars=true,
  breaklines=true,
  % German umlauts
  literate=%
  {Ö}{{\"O}}1
  {Ä}{{\"A}}1
  {Ü}{{\"U}}1
  {ß}{{\ss}}1
  {ü}{{\"u}}1
  {ä}{{\"a}}1
  {ö}{{\"o}}1
}
%




\graphicspath{{D:/GitHub/AS_MEI/relatorio/img/}}

\title{\textbf{Projeto da Unidade Curricular Arquiteturas de Sistema\\Mobilidade Urbana\\ MEI}}
				
\author{Jéssica Macedo a6835,\\ Fernando Correia a11199, \\Andreia Oliveira a12153}


\affil{2º Trabalho\\
      Escola Superior de Tecnologia\\
       Instituto Politécnico do Cávado e do Ave}	

\date{Barcelos, 15 de Dezembro de 2019}

\begin{document}

\maketitle
\tableofcontents


\listoftables


\chapter*{Introdução}
\addcontentsline{toc}{chapter}{Introdução}

O trabalho abordado no presente relatório foi desenvolvido no âmbito da unidade curricular Arquiteturas de Sistemas do mestrado em Engenharia de Sistemas Informáticos em Desenvolvimento de Aplicações. Tem como fundamental objetivo o desenvolvimento de um sistema distribuído que permite alugar veículos de mobilidade urbana, tendo por base uma API Restful que garante a integração entre a aplicação servidor e as várias aplicações cliente (um cliente agente, um cliente gestor, e um cliente dashboard).

\clearpage



\chapter*{Desenvolvimento}
\addcontentsline{toc}{chapter}{Desenvolvimento}

\section*{Problema a Resolver}
\addcontentsline{toc}{section}{Problema a Resolver}

Este capítulo aborda a descrição do problema e os seus objetivos.

\subsection*{Servidor}
\addcontentsline{toc}{subsection}{Servidor}

\begin{enumerate}
\item O sistema tem como objetivo agilizar o aluguer de veículos disponíveis, fornecendo:
\begin{itemize}
\item informação sobre os veículos livres;
\item filtros para a localização dos veículos livres;
\item gestão dos dados de cliente;
\item registo do pagamento através de um saldo recarregável.

\end{itemize}

\item O servidor deverá contemplar a utilização de bases de dados onde toda a informação
relacionada com o serviço disponibilizado será guardada;

\item Desenvolver um
conjunto de serviços para garantir o acesso à informação da base de dados, de forma a
responder aos pedidos dos diferentes clientes;

\item Disponibilizadar uma documentação (Open API) e descrição acerca dos testes realizados à utilização dos serviços;
\item Publicação num ambiente cloud dos diferentes serviços
desenvolvidos;
\item Seguir uma arquitetura baseada em micro serviços – containers;

\item Utilizar uma Gateway para facilitar a integração dos vários micro serviços;
\item  Disponibilizar um sistema integrado de logging global a todos os micro serviços.

\end{enumerate}
\clearpage


\subsection*{Utilizadores}
\addcontentsline{toc}{subsection}{Utlizadores}

Neste trabalho estão presentes quatro tipos de utilizadores, dos quais são:
\begin{itemize}
\item \textbf{Utilizador não registado} - Trata-se de um utilizador sem qualquer registo na plataforma;
\item \textbf{Cliente} - Trata-se de um utilizador previamente registado,
sendo considerado um cliente (do serviço de aluguer de veículos). Tem as mesmas funcionalidades que um utilizador não registado e mais algumas
para além deste.
\item \textbf{Funcionário} - Trata-se de um funcionário da entidade responsável pela gestão dos veículos,
que tem a responsabilidade de fiscalizar os estacionamentos.
\item \textbf{Administrador} - Trata-se da entidade fiscalizadora da aplicação. Consulta métricas, valida registos e configura.\\
\end{itemize}

\subsubsection*{Funcionalidades dos Utilizadores}
\addcontentsline{toc}{subsubsection}{Funcionalidades dos Utilizadores}
De seguida apresentamos as funcionalidades de cada utilizador:

\begin{enumerate}
\item \textbf{Utilizador não registado}

\begin{enumerate}
\item Permite obter informação dos lugares de estacionamento (latitude e longitude), capacidade, quantidade de veículos;
\item Permite registar-se e consequentemente autenticar-se na aplicação.

\end{enumerate}
\bigskip
\item  \textbf{Cliente}


\begin{enumerate}
\item Utilizador previamente registado
\item Permite obter informação dos lugares de estacionamento (latitude e longitude), capacidade, quantidade de veículos;
\item Permite pesquisar veículos detalhando o nome da rua ou raio de pesquisa
\item Consulta do saldo atual da conta
\item Fazer check-in do veículo (código veículo, método de aluguer [preço por minuto/pacotes de horas], hora inicio, preço estimado, código de aluguer)
\item  Fazer check-out do veículo (hora fim, verifica posição estacionamento, cálculo aluguer )
\item Fazer consulta dos dados relativos ao aluguer ativo (tempo e custo até ao momento)
\end{enumerate}
\bigskip
\item \textbf{Funcionário}
\begin{enumerate}
\item Registo de estacionamentos de veículos em locais impróprios
\item  Notificar cliente de estacionamento impróprio
\end{enumerate}
\clearpage
\item \textbf{Administrador}

\begin{enumerate}
\item Consultar dashboard com resumo dos dados e histórico de ocupação de lugares
\item  Permitir a validação do pedido de registo de utilizadores
\item Configuração da localização dos lugares de estacionamento
\item Nice to have: envio de indicação aos clientes da aproximação do fim do saldo
\end{enumerate}

\end{enumerate}


\clearpage

\section*{Plano de Desenvolvimento}
\addcontentsline{toc}{section}{Plano de Desenvolvimento}

Esta secção apresenta os objetivos propostos para as próximas entregas.

\begin{center}
\begin{table}[h!]
\begin{tabularx}{\textwidth}{@{} >{\hsize=0.41\hsize}L >{\hsize=1.65\hsize}L l l @{}} 
   %% Note: 0.65+1.35 = 2 = # of columns of type X
\hline
\textbf{Data} & \textbf{Objetivos}  \\
\hline
\multirow{1}{9em}{15/12/2019} & Geração dos modelos de dados
  \\ 

\hline
\multirow{3}{9em}{5/01/2020} & Geração dos serviços CRUD para os diferentes serviços  \\ 
& Criação da documentação Swagger\\ 
& Criação da aplicação frontend em React \\ 
\hline
\multirow{2}{9em}{17/01/2020} & Instalação do sistema em serviço cloud com o Heroku\\ 
& Continuação da criação da aplicação frontend\\ 
 
\hline
\end{tabularx}
\caption{Objetivos do Plano de Desenvolvimento}
\label{table:1}
\end{table}
\end{center}

\clearpage



\section*{Modelo de Dados}
\addcontentsline{toc}{section}{Modelo de Dados}

Nesta secção apresentamos o modelo de dados definido.

\begin{enumerate}
\item \textbf{Vehicle}\\

De seguida é apresentado o código para construir o schema do modelo de dados do Veículo.
\bigskip
\begin{lstlisting} [style=htmlcssjs]

var vehicleSchema = new Schema({
    code: {
        type: Number,
        required: [true,'code of the vehicle']
    },
    description: {
        type: String
    },
    place: [placeSchema]
});
\end{lstlisting}
\clearpage
\item \textbf{Client}\\
De seguida é apresentado o código para construir o schema do modelo de dados do Cliente.
\bigskip


\begin{lstlisting} [style=htmlcssjs]
var clientSchema = new Schema({
    firstName: {
        type: String,
        required: 'first name of the person'
    },
    lastName: {
        type: String,
        required: 'last name of the person'
    },
    rentals: [rentalSchema],
    balance: {
        type: Number,
    },
    registerBy: {
        type: mongoose.Schema.Types.ObjectId,
        ref: 'User'
    },
    Created_data: {
        type: Date,
        default: Date.now
    }
});
\end{lstlisting}
\clearpage
\item \textbf{User}\\
De seguida é apresentado o código para construir o schema do modelo de dados do Utilizador.
\bigskip


\begin{lstlisting} [style=htmlcssjs]
var userSchema = new Schema({
    username: {
        type: String,
        unique: true,
        required: true
    },
    email: {
        type: String,
        unique: true,
        index: true,
        required: true
    },
     password: {
type: String,
 	required: true,
select: false
    },
role:{ 
type: String,
 required: true, 
default: 'client , 
enum: ['client','employee','admin']
}
});

\end{lstlisting}
\clearpage
\item \textbf{Rental}\\
De seguida é apresentado o código para construir o schema do modelo de dados do Aluguer.
\bigskip
\begin{lstlisting} [style=htmlcssjs]
var rentalSchema = new Schema({
    startDate: {
        type: Date,
        // default: Date.now
    },
    endDate: {
        type: Date,
        // default: Date.now
    },
    status: {
            type: String,
            enum: ['confirmed', 'canceled'],
               default: ['confirmed']
    },
    price: {
        type: Number,
        required: true
    },
paymentMethod:{
type: String,
            enum: ['minutes', 'pack'],
             default: ['minutes']
},
    code: {
        type: Number,
        required: true
    },
    vehicle: [vehicleSchema]
});

\end{lstlisting}
\clearpage
\item \textbf{Place}\\
De seguida é apresentado o código para construir o schema do modelo de dados do Lugar.
\bigskip

\begin{lstlisting} [style=htmlcssjs]
var placeSchema = new Schema({
    location: {
        type: String,
        coordinates: [Number],
        required: true
    },
    capacity: {
        type: Number
    },
    quantity: {
        type: Number,
    }
});

\end{lstlisting}
\end{enumerate}
	


\clearpage





\clearpage

\begin{thebibliography}{5}
\bibitem{rep}
		 Repositório, \url{https://github.com/Knox316/MobilityProject}
	\bibitem{react}
		React, \url{https://reactjs.org/}.
	\bibitem{heroku}
		 Heroku, \url{https://www.heroku.com/}
	





	
\end{thebibliography}

\clearpage

\end{document}